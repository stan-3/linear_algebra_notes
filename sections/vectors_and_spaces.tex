\documentclass[../main.tex]{subfiles}


\begin{document}

\subsection{Parametric representations of lines}

\begin{example}
	Suppose that $L_1$ and $L_2$ are lines in the plane, that the
	x-intercepts of $L_1$ and $L_2$ are 5 and -1, respectively, and that the
	respective y-intercepts are 5 and 1. Find the point of intersection
	of $L_1$ and $L_2$.
\end{example}

\begin{solution}
	Pick two points on $L_1$,
	i.e. $(5, 0)$ and $(0, 5)$. Let $\vect{a}=\langle 5,\ 0\rangle$ and $\vect{b}=\langle 0,\ 5\rangle$.
	Now, $L_1$ can be represented by the following:

	\begin{equation*}
		\begin{split}
			L_1 = & \{ (\vect{a} - \vect{b})t + \vect{b} \mid t \in \mathbb{R} \}                            \\
			=     & \left. \left\{ \begin{bmatrix} 5t\\-5t+5 \end{bmatrix} \right| t \in \mathbb{R} \right\}
		\end{split}
	\end{equation*}

	The $(\vect{a} - \vect{b})t$ represents $L_1$ parameterized through the origin,
	so we apply a translation of $\vect{a}$ or $\vect{b}$. Similarly,
	let $\vect{c}= \langle -1, 0 \rangle$ and $\vect{d}= \langle 0, 1 \rangle$.

	\begin{equation*}
		\begin{split}
			L_2 = & \{ (\vect{c} - \vect{d})s + \vect{d} \mid s \in \mathbb{R} \}                             \\
			=     & \left. \left\{ \begin{bmatrix} -s \\ -s+1 \end{bmatrix} \right| s \in \mathbb{R} \right\}
		\end{split}
	\end{equation*}

	The point of intersection is where $L_{1_x}=L_{2_x}$ and $L_{1_y}=L_{2_y}$,
	allowing us to define a system of equations.

	\begin{align*}
		5t= & -s & -5t+5=-s+1 \\
	\end{align*}

	\begin{align*}
		s= & -2     \\
		x= & -s=2   \\
		y= & -s+1=3
	\end{align*}

	The point of intersection is (2, 3). $\square$
\end{solution}

\subsection{Linear dependence}

\begin{definition}[Linear combination]
	Let $V=\{ \vect{v_1}, \vect{v_2}, \ldots , \vect{v_k} \}$ where $\vect{v_i} \in \mathbb{R}^n$.
	A \textit{linear combination} of $V$ is defined to be
	$c_1\vect{v_1} + c_2\vect{v_2} + \dots + c_k\vect{v_k}$ where $c_i \in \mathbb{R}$.
\end{definition}

\begin{definition}[Linearly dependent]
	Let $V=\{ \vect{v_1}, \vect{v_2}, \ldots , \vect{v_k} \}$ where $\vect{v_i} \in \mathbb{R}^n$.
	$V$ is \textit{linearly dependent} if and only if
	$c_1\vect{v_1} + c_2\vect{v_2} + \dots + c_k\vect{v_k} = \vect{0}$
	where $c_i \in \mathbb{R}$ and there exists at least one $c_i$
	such that $c_i \neq 0$. In other words, $V$ is \textit{linearly independent}
	if and only if $c_1,c_2,\ldots ,c_k=0$ is the only solution.
\end{definition}

\begin{definition}[Span]
	Given a set $S$ of vectors, the \textit{span} of S,
	denoted $\operatorname{span}(S)$, is the set of all linear combinations of $S$.
\end{definition}

\begin{example}
	Let $S=\left\{
		\begin{bmatrix} 1 \\ -1 \\ 2 \end{bmatrix},
		\begin{bmatrix} 2 \\ 1 \\ 3 \end{bmatrix},
		\begin{bmatrix} -1 \\ 0 \\ 2 \end{bmatrix}
		\right\}$. Is $S$ linearly dependent?
\end{example}

\begin{solution}
	\begin{equation*}
		\begin{bmatrix} 1 \\ -1 \\ 2 \end{bmatrix}c_1
		+ \begin{bmatrix} 2 \\ 1 \\ 3 \end{bmatrix}c_2
		+ \begin{bmatrix} -1 \\ 0 \\ 2 \end{bmatrix}c_3 = 0
	\end{equation*}

	\begin{equation*}
		\begin{split}
			c_1+2c_2-c_3=   & \ 0 \\
			-c_1+c_2=       & \ 0 \\
			2c_1+3c_2+2c_3= & \ 0
		\end{split}
	\end{equation*}

	Solving this system, we find that the only solution is
	$c_1=c_2=c_3=0$. Thus, $S$ is linearly independent. $\square$

	We can go further by saying that the span of $S$ is $\mathbb{R}^3$. This
	can be proven by setting the linear combination of $S$ to an
	arbitrary 3-dimensional vector and isolating for the scalars $c_1$, $c_2$, and $c_3$,
	which tells us that any given vector in $\mathbb{R}^3$ can be represented
	in a specific linear combination of $S$. This can be thought of intuitively as well:
	if $S$ is linearly independent, each vector of $S$ introduces
	new directionality; there is no vector in $S$ that can be defined as a linear
	combination of the other vectors.
\end{solution}

\subsection{Linear subspaces}

\begin{definition}[Linear subspace]
	A set of vectors $V \subseteq \mathbb{R}^n$ is defined to be a linear/vector \textit{subspace}
	of $\mathbb{R}^n$ if and only if it contains $\vect{0}$,
	it is closed under scalar multiplication, and it is closed under addition:

	\begin{align*}
		\vect{0} \in                                                       & V \\
		\forall\ c \in \mathbb{R},\ c\vect{v} \in                          & V \\
		\forall\ \vect{v_1}, \vect{v_2} \in V, \vect{v_1} + \vect{v_2} \in & V
	\end{align*}
\end{definition}


\begin{theorem}
	Let $V=\{ \vect{v_1}, \vect{v_2}, \ldots , \vect{v_n} \}$. It holds true that
	$\operatorname{span}(V)$ is a subspace of $\mathbb{R}^n$.
\end{theorem}

\begin{proof}
	If $\operatorname{span}(V)$ is a valid linear subspace, it must contain the zero vector,
	it must be closed under scalar multiplication, and it must be closed
	under addition.

	By definition, the span of $V$ is the set of all linear combinations of $V$:

	\begin{equation}
		\operatorname{span}(V)=\{ c_1\vect{v_1} + c_2\vect{v_2} + \dots + c_n\vect{v_n}\ \forall\ c_i \in \mathbb{R} \}
	\end{equation}

	1. Inclusion of zero vector:

	\begin{center}
		Let $c_1, c_2, \ldots , c_n=0$.
		$$\implies c_1\vect{v_1} + c_2\vect{v_2} + \dots + c_n\vect{v_n}=\vect{0}$$
		$$\implies \vect{0} \in \operatorname{span}(V)$$
	\end{center}

	2. Closure under scalar multiplication:

	\begin{center}
		Let $\vect{a} \in \operatorname{span}(V)$.
		$$\iff c_1\vect{v_1} + c_2\vect{v_2} + \dots + c_n\vect{v_n}=\vect{a}$$
		$$d(c_1\vect{v_1} + c_2\vect{v_2} + \dots + c_n\vect{v_n})=d\vect{a}\ \textnormal{where}\ d \in \mathbb{R}$$
		$$\implies dc_1\vect{v_1} + dc_2\vect{v_2} + \dots + dc_n\vect{v_n}=d\vect{a}$$
	\end{center}

	$dc_i$ is just a scalar, meaning $d\vect{a}$ is another
	linear combination of $V$:
	$$d\vect{a} \in \operatorname{span}(V)$$

	3. Closure under addition:

	\begin{center}
		Let $\vect{a}, \vect{b} \in \operatorname{span}(V)$.
		$$c_1\vect{v_1} + c_2\vect{v_2} + \dots + c_n\vect{v_n} = \vect{a}$$
		$$d_1\vect{v_1} + d_2\vect{v_2} + \dots + d_n\vect{v_n} = \vect{b}$$
		$$(c_1+d_1)\vect{v_1} + (c_2+d_2)\vect{v_2} + \dots + (c_n+d_n)\vect{v_n} = \vect{a} + \vect{b}$$
	\end{center}

	Again, $(c_i+d_i)$ is just a scalar, meaning $\vect{a}+\vect{b}$ is another linear combination of $V$:
	$$\vect{a}+\vect{b} \in \operatorname{span}(V)$$
\end{proof}

\begin{definition}[Basis]
	Let $S = \{ \vect{v_1}, \vect{v_2}, \ldots , \vect{v_n} \}$ be linearly independent.
	It follows that S is a \textit{basis} for the subspace $V = \operatorname{span}(S)$.
\end{definition}

\end{document}
